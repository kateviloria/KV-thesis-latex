\documentclass[11pt, a4paper]{article}

\usepackage{mlt-thesis-2015}

% With Xetex/Luatex this shouldn't be used
%\usepackage[utf8]{inputenc}

\usepackage[english]{babel}
\usepackage{graphicx}
\usepackage{caption}
\usepackage{subcaption}

\usepackage{setspace}
\usepackage{subfiles} 
\usepackage{hyperref}
\usepackage{booktabs}
%\graphicspath{ {figures/} }

\hypersetup{
colorlinks,
allcolors={blue}
}


\title{Wanna Be On Top?}
\subtitle{The Hyperparameter Search for Semantic Change's Next Top Model}
\author{Kate Viloria}

\begin{document}

%% ============================================================================
%% Title page
%% ============================================================================
\begin{titlepage}

\maketitle

\vfill

\begingroup
\renewcommand*{\arraystretch}{1.2}
\begin{tabular}{l@{\hskip 20mm}l}
\hline
Master's Thesis: & 30 credits \\
Programme: & Master’s Programme in Language Technology\\
Level: & Advanced level \\
Semester and year: & Spring, 2021\\
Supervisor: & Simon Hengchen\\
Examiner: & Eleni Gregoromichelaki\\
Report number: & (number will be provided by the administrators) \\
Keywords: & semantic change, language change, diachronic word embeddings
\end{tabular}
\endgroup

\thispagestyle{empty}
\end{titlepage}

%% ============================================================================
%% Abstract
%% ============================================================================
\newpage
\singlespacing
\section*{Abstract}

RE-READ


Lexical semantic change (LSC) detection through the use of diachronic corpora and computational methods continues to be a prevalent research area in language change. However, there has not been extensive work on further examining the models being trained and creating a foundation for what hyperparameter settings yield the best results.  In this thesis,  a large-scale hyperparameter search is conducted using the SemEval-2020 Task 1 dataset that includes English, German, Swedish, and Latin. Alongside model hyperparameters, algorithms (Word2Vec and FastText) and alignment methods (Orthogonal Procrustes and Incremental Training) were also included. The hyperparameters that were evaluated are: epochs, vector dimension size, frequency threshold, and shared vocabulary size for the Orthogonal Procrustes alignment method. By amalgamating all of the results and seeing how model performances are affected if one hyperparameter is changed, considerations that must be made before training a model were substantiated. Conclusions made were that there is a large decrease in performance improvement rate after 50 epochs in training and that the typical choice of 300 dimensions for vectors (based on English LSC tasks) does not necessarily apply to other languages. It is also shown that choices in dimension size, frequency threshold, and shared vocabulary size depend on the language in question, the corpus' size and text genre composition. 

\thispagestyle{empty}

%% ============================================================================
%% Preface
%% ============================================================================
\newpage
\section*{Preface}

Acknowledgements, etc.

\thispagestyle{empty}

%% ============================================================================
%% Contents
%% ============================================================================
\newpage

\begingroup
\hypersetup{linkcolor=black} % This ensures the ToC has black links
\begin{spacing}{0.0}
\tableofcontents
\end{spacing}
\endgroup

\thispagestyle{empty}

%% ============================================================================
%% Introduction
%% ============================================================================
\newpage
\setcounter{page}{1}



\section{Introduction}
\label{sec:intro}

Neural embeddings have had a large rise in popularity as the go-to method for modelling diachronic text. However, these neural models are not critically evaluated and continue to overlook many of the aims and questions of language change \citep{hengchen2021challenges}. The release of the SemEval-2020 Task 1 on Unsupervised Detection of Lexical Semantic Change (LSC) \citep{schlechtweg-etal-2020-semeval} "laid the groundwork for the first comparative study of computational models of LSC" in English, German, Latin, and Swedish \citep{hengchen2021SBXrushifteval}. There are also no foundations or baselines for what hyperparameter variables yield the highest accuracy for this domain-specific task. However, it has been proven that hyperparameter searches for tasks in LSC are worthwhile and produced large improvements in performance (\citet{kaiser-etal-2020-ims, hengchen2021SBXrushifteval}). This thesis presents the results to a large-scale hyperparameter search using the SemEval-2020 Task 1 corpora. The effects of changes in hyperparameter variables and how the hyperparameters interact with the data are also discussed. Intuitions for future tasks in LSC are proposed and considerations toward corpora being used, language being investigated, and target words (frequency and part-of-speech (POS) tag) for evaluation are put forth. 


\subsection{Research Questions}
%\cmtSH[inline]{This can be higher level. You don't care about SemEval 2020 Task 1, you care about language. But there's no test set for *LANGUAGE*, so you take the next best thing: SemEval 2020 Task 1.}
%\cmtSH{This thesis :)} 
This thesis aims to find and investigate the top-performing models for detecting LSC and examine the relationships between different hyperparameter variables and model performance. This will be done through building a pipeline that will train and evaluate 4000 hyperparameter combinations. It also aims to present considerations regarding data, language, and target words that must be made before training models for LSC by extrapolating from the results collected. 


\subsection{Motivation and Real-World Application}

As with many approaches where linguistics and computational methods intersect, “computational models of word meaning are often taken at face value and not questioned by researchers working on LSC” \citep{hengchen2021challenges}. Evaluating models not only in their performance but also grounded with reasoning based on linguistic theory is crucial. Since models are language-specific and time-specific, assessments in the field of LSC must also be analysed with linguistic and sociological lenses. Through conducting a hyperparameter search, all other variables are controlled and models can be thoroughly evaluated and analyzed through the two lenses mentioned above. 

\citet{hengchen2021challenges} also state that there, understandably so due to the large availability and resources put into creating English corpora, has been a large over-representation of studies performed on English. Although there have been great advances for the English language in this field, these tools and methods are not necessarily transferable or applicable to other languages. By using the SemEval dataset, languages other than English are being represented and evaluated. 

Advancements in the field of LSC would be beneficial to many other fields and real-world applications. LSC approaches can be used by lexicographers in identifying and validating the usages and dates of usage of specific word senses \citep{lau-etal-2012-word, falk-etal-2014-non, klosa-2018-newgerman}. The field of historical linguistics would also benefit from these approaches in order to test the different laws or hypotheses involving how languages change \citep{hamilton-etal-2016-diachronic, Xu2015ACE}. It is also mentioned in \citet{hengchen2021challenges} that these methods are transferable to other fields such as the interpretation of literature in historical research and political science.  

%\cmtKV{is below the better way of citing multiple papers within parentheses}
%(\citet{lau-etal-2012-word,falk-etal-2014-non,klosa-2018-newgerman})\\
%\cmtSH{Here below works better if they need to be within parentheses -- it removes an extra set of ()}
%\citep{lau-etal-2012-word,falk-etal-2014-non,klosa-2018-newgerman}

\subsection{Contributions}
In this thesis, a hyperparameter search tool is built and used to conduct the large-scale hyperparameter search and set a precedence for intuition when choosing hyperparameters for future LSC tasks. Through training various hyperparameter combinations, trends in performance are analysed and recommendations for hyperparameters are substantiated. Important considerations that must be made before training a model for an LSC task include: the language in question, corpora size and text genres, and the frequency and POS-tags of the target words. Performance evaluation of the models by POS-tag showed that some models are better at detecting LSC in certain POS-tags than others. Having the groundwork for which hyperparameter combinations most accurately detects LSC for specific languages and corpora will allow the research area to focus on discerning when a word gains or loses a sense and what those senses are. 


\subsection{Scope}
\label{intro-scope}
Distributional approaches face many criticisms in the field of computational semantics. Through an evaluation of their novel approach that considers syntactic relations and traditional vector-based models, \citet{pado-lapata-2003-constructing} note that traditional semantic space models have a more difficult time differentiating semantic relations between word pairs. Models used today for detecting semantic change still have a difficult time discerning the type of semantic relation or change that is occurring in a vector space. Evaluating the specification of semantic relations and changes between words is not within the scope of this thesis. Performance of the models are based on whether or not a model can accurately detect whether a word has undergone semantic change. The ability to discern whether or not a target word has gained or lost a sense is not evaluated. 



%% TODO:
% write motivation
% proofread subsection 3
% etc

%% ALREADY DONE:
% write xyz
% fixed bibtex
% etc

%% You can add and edit these comments as you see fit for the other sections, or use some other tool

\newpage

\section{Background}
% subsections are \subsection{title}
%% subssubsections are \subsubsection{title}
%% numbering will work automatically

\subsection{2.1 - Distributional Hypothesis}

The Distributional Hypothesis theoretically drives the current and leading models for detecting LSC. The rationale being that “there is a correlation between distributional similarity and meaning similarity, which allows us to utilize the former in order to estimate the latter”—in simpler and more familiar terms, “words which are similar in meaning occur in similar contexts.” \citep{sahlgren2008distributional} The distributional methodology presented in \citet{harris1970distributional} is built on structuralist theory. A structuralist approach to language focuses on the general construction of a language system rather than the idiolectal use of language. According to \citet{sahlgren2008distributional}, Saussure identifies the functional differences of linguistic meaning into syntagmatic and paradigmatic relations. Syntagmatic relations involve the syntactic positioning or sequence of words. The combination and order of linguistic entities form a syntagmatic relationship that then creates meaning. Paradigmatic relations are between words that appear in the same context but do not co-occur. Given these characteristics, linguistic entities that have a paradigmatic relationship should be interchangeable within the same context or sentence. \citet{sahlgren2008distributional} offers the refined distributional hypothesis as “A distributional model accumulated from co-occurrence information contains syntagmatic relations between words, while a distributional model accumulated from information about shared neighbors contains paradigmatic relations between words.” With this in mind, models with a larger context window are more likely to detect or learn paradigmatic relations. Depending on the manipulation of specific model hyperparameters, certain relations will be learned. 


\subsection{2.1 - Lexical Semantic Change(??)}
LSC detection through computational methods still face many challenges today. \citet{hengchen2021challenges} have identified two approaches in the computational field of LSC—treating a word as an entity and determining semantic change based on its dominant sense and treating each word’s sense as a separate entity. Both approaches however, mostly capture contextual similarity between lexical items while the different levels of meaning are seldom distinguished \citep{hengchen2021challenges}. There are three different methods to model the meaning of a word computationally: each word in the vocabulary and all its semantic information will have one representation (e.g., static embeddings), each word being split into different semantic areas resulting in representations that approximate a word’s senses (e.g., topic modelling), and each word having a representation for every time it is used in a sentence (e.g., contextual embedding). Each method can be successful given it is applied appropriately in different tasks and what kind of LSC problem is trying to be solved. It is also important to note that not all approaches are able to model or differentiate the different senses a word could possess. In addition to the word sense limitations these representations have, count representations have also been proven to introduce an inherent dependence on word frequency—resulting in random noise in the models \citep{dubossarsky-etal-2017-outta}. By comparing different corpora with each other, original and shuffled, \citet{dubossarsky-etal-2017-outta} demonstrates that there is a strong correlation between the change scores of words and their frequencies. 

Once words have been created into representations, there are also different calculations that can be used when comparing word representations between two time periods (e.g. cosine distance, Euclidean distance, Jensen-Shannon divergence, etc.). With these calculations, there are two ways to evaluate large datasets and tasks involving the detection of LSC—binary LSC (i.e. has a word changed or not) or a graded LSC (i.e. to what degree has a word changed). Although these approaches present a systematic way of evaluating the current models of meaning being created, the question of what kind of change in meaning has a word undergone remains unanswered.

Corpora and datasets used in the field of LSC are largely in English. However, to assume that semantic change occurs in similar ways across other languages is extremely incorrect. As stated by \citet{bender_2020}, the advancements in NLP (ACRONYM) rely on the existence of language resources and English is neither synonymous with nor representative of “Natural Language”. English, as a high resource language, naturally results in more research being published on English. The field of LSC is no different. Apart from a handful of corpora in different languages (\citet{diacrita_evalita2020} for Italian and \citet{rushifteval2021} for Russian), most research in LSC are conducted on the English language. In order to study semantic change effectively and successfully, the variation that already exists between languages must be considered. The creation of resources for languages other than English is crucial to the development of LSC and in turn (NOT USING RIGHT?), NLP.   

\newpage

\section{Experimental Setup}

\subsection{3.1 - Dataset}

The SemEval-2020 Task 1 on Unsupervised Detection of Lexical Semantic Change \citep{schlechtweg-etal-2020-semeval} provides manually-annotated datasets for four languages (English, German, Swedish, and Latin). It is composed of a diachronic corpus pair and a set of target words for each language. Having a gold standard that is based on roughly 100,000 human judgments, researchers now have a more concrete foundation for comparing models. There are two subtasks that differ by the assessment of LSC: binary classification and ranking. In order to identify and evaluate the subtle effects of hyperparameter changes, this thesis follows Subtask 2—based on a ranking of the target words depending on the degree of LSC between the first and second corpus. In contrast to Subtask 1’s binary classification, the method of ranking in Subtask 2 “captures fine-grained changes in the two sense frequency distributions” \citep{schlechtweg-etal-2020-semeval}. The two corpora for each language (C1 and C2) were divided based on data size and the availability of target words. Pre-processing of all corpora involved lemmatization, removal of all punctuation, and randomly shuffling sentences within each time-specific corpora.

The Clean Corpus of Historical American English (CCOHA) consists of different types of text—fiction, non-fiction, magazines, and newspapers—from the 1810s to the 2000s \citep{davies2012expanding, alatrash-etal-2020-ccoha}. The first time-bin corpus used for German is Deutsches Textarchiv (DTA) which is composed of different genres of text from the 16th-20th centuries \citep{dta2017}. For the second time-bin, a combination of the Berliner Zeitung (BZ) and Neues Deutschland (ND) corpora is used \citep{berliner2018,neues2018}. Both German corpora are comprised of newspaper articles from the years of 1945-1993. The corpus used for Latin, LatinISE \citep{mcgillivray-kilgarriff}, is a compilation of texts originating from 2nd century B.C. to 21st century AD from three online digital libraries. The corpora used for Swedish is the Kubhist corpus \citep{Kubhist}, which similar to the German corpus used, is a newspaper corpus with articles from the 18th to the 20th century. \hfill \break
\begin{table}[h]
\small
\centering
\begin{tabular}{l|cc|cc|}
\cline{2-5}
\textbf{}      & \multicolumn{2}{c|}{\textbf{$C_1$}}                    & \multicolumn{2}{c|}{\textbf{$C_2$}}                    \\
                                       & \textit{\textbf{corpus}} & \textit{\textbf{period}} & \textit{\textbf{corpus}} & \textit{\textbf{period}} \\ \hline
\multicolumn{1}{|l|}{\textbf{English}} & CCOHA                    & 1810-1860                & CCOHA                    & 1960-2010                \\ \hline
\multicolumn{1}{|l|}{\textbf{German}}  & DTA                      & 1800-1899                & BZ + ND                  & 1946-1990                \\ \hline
\multicolumn{1}{|l|}{\textbf{Latin}}   & LatinISE                 & -200-0                   & LatinISE                 & 0-2000                   \\ \hline
\multicolumn{1}{|l|}{\textbf{Swedish}} & Kubhist                  & 1790-1830                & Kubhist                  & 1895-1903                \\ \hline
\end{tabular}
\caption{Time periods of each sub-corpora for each language.}
\label{tab:subcorpora-time}
\end{table}
\hfill \break
Annotators—all native speakers or former university students of the respective language they were assigned—were instructed to follow the DURel framework \citep{DURel2018}. Deriving from \citet{blank1997prinzipien}’s continuum of semantic proximity for synchronic polysemy annotation, its semantic-relatedness scale for a target word \emph{w} within two specific time periods from C1 and C2 resulted in high inter-annotator agreement. 
	
Each language is accompanied by a target word list that consists of words that have undergone semantic change or stable words—words that have not changed in meaning. For words that have changed in meaning, it is not distinguished whether words have lost or gained a sense. Stable words were chosen to act as counterparts of words that have changed in meaning through the consideration of the same POS tag and a comparable/similar frequency development between the two time periods. This consideration minimizes possible model biases that result from these factors. \citep{dubossarsky-etal-2017-outta} Since many of the English target words underwent POS-specific semantic changes, POS tags have been concatenated in the target word list (“word\_pos”). Although the addition of POS tags in the SemEval task was for the purpose of English words having a tendency to change POS tags when changing senses, it was also a great opportunity to examine model performances based on POS tags for this project. The addition of POS tags for the target word lists of the remaining three languages will be discussed further below.
 
% Please add the following required packages to your document preamble:
% \usepackage[table,xcdraw]{xcolor}
% If you use beamer only pass "xcolor=table" option, i.e. \documentclass[xcolor=table]{beamer}
\begin{table}[h]
\small
\centering
\begin{tabular}{|c|c|c|c|}
\hline
\textbf{Language} & \textbf{NN} & \textbf{VB} & \textbf{ADJ} \\ \hline
\textit{English}                          & 33          & 4           & 0            \\ \hline
\textit{German}                           & 32          & 14          & 2            \\ \hline
\textit{Latin}                            & 28          & 5           & 7            \\ \hline
\textit{Swedish}                          & 22          & 6           & 3            \\ \hline
\end{tabular}
\caption{Number of nouns (NN), verbs (VB), and adjectives (ADJ) in each language's target word list.}
\label{tab:postag-breakdown}
\end{table}

(FROM MOTIVATION)
\cmtKV[inline]{this section feels out of place but at the same time should be mentioned, just not sure if it's better off in motivation}
Another challenge that the field is currently facing is having a more robust system of evaluation. Currently, semantic annotation stands as the most reliable (sole (TECHNICALLY NOT)) way of evaluating and validating semantic change in historical corpora \citep{hengchen2021challenges}. Although effective (for now), the annotating process to obtain ground-truth data results in large amounts of money and time expended. It also only produces a limited target word list—allowing the possibility of evaluation inconsistencies to be introduced. A model might perform well with one target list and horribly with another. (CONJECTURE? LOL) Using a simulated LSC for evaluation is in its early stages and is encouraged to be used for evaluation in tandem with ground-truth testing.

Numerous types of semantic representations were used by teams during the SemEval task: token embeddings vs. type embeddings and topic modelling vs. vector space models. In both Subtask 1 and 2, the highest-performing systems used static-type embedding models. Token embeddings include more contextual information each time the target word appears in the corpus while type embeddings are average embeddings (LACKING). However, it was pointed out that raw corpora, rather than lemmatized corpora, would bode better for token embeddings. (ANNOTATION?) Following the best-performing models of the tasks, with a larger focus on Subtask 2, a hyperparameter search of models with type embeddings would help determine if the top scores of the task could be surpassed solely on changes in hyperparameters. (ENDSOUNDS WEIRD, EDIT) 

\subsection{3.X - Experimental Large-Scale Hyperparameter Search (TOO LONG?)}
In order to clarify the scope and justify the hyperparameter range for this large-scale hyperparameter search, the results from the SemEval-2020 Task 1 are used as the basis/starting point (PICK ONE). The top-performing models were taken into consideration along with the post-hoc improvements and analyses of those models by \citet{kaiser-etal-2020-ims}. The pipeline was coded and structured with the possibility of reusing the code for conducting similar hyperparameter searches in future LSC tasks with the hopes of the hyperparameter searches to be based on the findings of this thesis—and in turn, hyperparameters with smaller ranges and motivated by linguistic and domain-specific reasoning. 

\cmtKV[inline]{**formatting, also think it would be useful to show the range after each subsubsection heading or would that be redundant, this below could also be neater, need to look into it}
The initial hyperparameters examined and analysed are shown below:
\begin{itemize}
    \item language $\in$ [English, German, Latin, Swedish],
    \item algorithm $\in$ [Word2Vec, FastText],
    \item alignment method $\in$ [Orthogonal Procrustes, Incremental Training],
    \item epochs $\in$ $[5, 10, 20, 50, 100]$,
    \item dimensions $\in$ $[10, 25, 50, 100, 300]$,
    \item frequency threshold $\in$ $[5, 10, 50, 100]$.
\end{itemize}%\vspace{+1ex}


**READABILITY SAKE, not canonically hyperparameters but are included when mentioning combination of hyperparameter

\subsubsection{Algorithm}
Type embeddings were used in the top-performing models for all four languages in Subtask 2.  Since type embeddings overwhelmingly outperformed token embeddings, static-type embeddings were chosen to be implemented. The same is seen in the hyperparameter search conducted by \citet{hengchen2021SBXrushifteval} with Russian data. The skip-gram negative sampling (SGNS) algorithm was used by the top 4 teams whose models performed well in Subtask 2 of the SemEval task. Bidirectional Encoder Representations from Transformers (BERT) was not used as \citet{laicher-2020} demonstrated that it resulted in poor results for an LSC task in Italian. Variety is provided in the algorithm hyperparameter by comparing the use of n-grams of Word2Vec and FastText. Word2Vec has shown success in LSC tasks and the use of FastText would address the morphological considerations for words that should be made as shown by \citet{bojanowski2017enriching}.

\subsubsection{Alignment Method}
The two alignment methods used are Orthogonal Procrustes and incremental training. Orthogonal Procrustes is implemented by using the first time period model as the base and the second time period model is “stretched” to fit the first. This is motivated by the fact that language builds on itself. In order to make comparisons of words changing over time, new usages of already existing words and new words (second time period model) should be made to fit into the contexts and usages that it proceeds (first time period model). The Gensim Word2Vec Procrustes Alignment tutorial by Ryan Heuser\footnote{\url{https://gist.github.com/quadrismegistus/09a93e219a6ffc4f216fb85235535faf}}  was initially used but was not compatible with FastText models. These aligned models were still saved and evaluated. However, all models were then aligned following the implementation of Team UWB @ SemEval \citet{prazak-etal-2020-uwb}’s model implementations using VecMap \footnote{\url{https://github.com/artetxem/vecmap}} by \citet{artetxe2018generalizing}. However, implementation differed from \citet{prazak-etal-2020-uwb}’s since the alignment chosen was supervised due to computational requirements and preliminary tests showed good performance. After seeing the results of VecMap’s supervised alignment method, another hyperparameter was introduced for models that are being aligned through Orthogonal Procrustes. There was an option on how many words can be provided in the training dictionary where the word vectors can be aligned automatically. Given a word \emph{w} from the first time period model, its equivalent to the second time period model would also be w. The additional hyperparameter was the amount of words given to VecMap as equivalents—making aligning the two models “easier” or “more supervised” (WORDING).

Models that have Orthogonal Procrustes as the alignment method will also have this added hyperparameter:
\begin{itemize}
    \item vocabulary size $\in$ $[1000, 5000, 10000, ALL]$
\end{itemize}%\vspace{+1ex}

The second alignment method, incremental training, was implemented by taking the first time period model and continue its training but this time using the second time period corpus. The models being compared will be the model trained on the first time period corpus and the model trained on both the first and second time period corpus.

\subsubsection{Epochs} \label{exp-epochs}
Epochs chosen were initially a very wide range from 5 to 100. A smaller preliminary test for higher epochs were conducted while training Word2Vec to see if there were significant improvements in performance and if it was worth exploring further. It was decided that FastText models trained on 100 epochs should not be pursued further due to computational costs and no added performance. The litmus test was done on Word2Vec models as it will always have less vectors since it operates on a per-word basis rather than FastText which creates vectors based on character \emph{n}-grams. 

\subsubsection{Dimensions}
Variety in dimensions were included in this hyperparameter search due to \citet{kaiser-etal-2020-ims}’s post-hoc analysis of the top-performing models in the SemEval-2020 Task 1 demonstrating that the same models can be outperformed by optimising *vector initialization alignment* (INCLUDE?) dimensionality. Their results also show that frequency-induced noise is introduced through vector space alignment and it is strongly correlated to dimensionality. With higher dimensions, vector representations could learn more specific contexts and usages of the word. However, high vector dimensions may also lead to more noise being introduced in the word representations. 

\subsubsection{Frequency Threshold}
Frequency thresholds can have a significant effect on computational time as well as overall performance. The frequency threshold is the amount of times a word has to appear in a corpus for it to be included in the training of the model. This threshold can either help eliminate words that are not relevant and only introduce noise but it can also eliminate important contexts that could help with the nuanced usages of words, possibly a target word. Exploring this hyperparameter would allow us to see if this hyperparameter has an optimal number when combined with other hyperparameters. 

\subsection{3.X - Evaluation/Spearman's Rank Order Correlation}
For SemEval-2020 Subtask 2 where LSC is quantified as a measurement of change through cosine distances, Spearman’s Rank Order Correlation is calculated as the main statistical measure of overall score. Once the cosine distances are calculated for each target word, they are ranked and compared with the gold truth labels. Spearman's correlation $\rho$ is then calculated along with its p-value. Spearman’s $\rho$ can be any value from -1 to 1 where 1 is a perfect positive correlation between ranks, -1 is a perfect negative correlation between ranks, and 0 is no correlation between ranks. In this task, the best possible score for a model would be 1, meaning that it has ranked the cosine distances exactly like the truth labels. Calculating through ranking rather than direct comparisons (such as the difference between each target word’s cosine distance) was chosen since it allows vector spaces to vary more in size given the large amount of hyperparameter combinations being implemented. (NOT SURE IF THIS IS TRUE OR JUST WORDED BADLY) The same evaluation will be done for each model but with the target words divided by POS-tag. Each model will have a Spearman’s $\rho$ for nouns, verbs, and adjectives. 

\subsection{3.X - Ethical Considerations and Efficiency Adjustments}
Initially, 1600 models were planned to be trained for this large-scale hyperparameter search. However, with the deduction of 100-epoch models for FastText and the addition of the vocabulary alignment hyperparameter for the Orthogonal Procrustes alignment method, 4000 models were trained in total. Throughout the training process, it was important to vigilantly take into consideration computational times, storage, and energy usage. As stated above in EPOCHS SECTION, 100-epoch models were trained on Word2Vec first and examined to see if the increase in epochs and therefore computational time and energy used also reflected in the model’s performance. Since it did not show promising results or results that increased in conjunction with time and energy spent, training 100-epoch models for FastText was deemed not worth pursuing further. 

The longer period of planning the pipeline was to ensure that once training begins, there are as little road blocks as possible. This would ensure that the training process and the storing of models were always organised and never lost or having to be retrained to avoid unnecessary consumption of resources. In hopes of remaining vigilant and conscious of the large amount of energy, resources, and storage being used for this thesis, adjustments to help with time and energy consumption were constantly being made to the pipeline. After the large-scale pipeline and the Orthogonal Procrustes alignment was incompatible with FastText models, the pipeline was restructured and already existing models were reformatted for efficiency purposes. Before the reformatting, 1TB of data and models were being stored. They were then transferred and reformatted as text files. To save computational time and energy, the cosine distances for the target words of each model were stored as text files and will be made available to the public along with the model text files and pipeline. This was incredibly helpful since the initial plan was for the model to iterate through the entire model to search for the target words and calculate the overall Spearman’s $\rho$ and again for nouns, verbs, and adjectives. Another adjustment made during training was for the incremental alignment method. The model trained on the first time period corpus with the same hyperparameter combinations but for the Orthogonal Procrustes alignment method was used and trained again with the second time period corpus. This eliminated training what would have been 720 duplicate models.  

This large undertaking was done using an Intel Xeon CPU E5-2640 v4 (25M Cache, 2.40 GHz). With about X hours total of training and computational time, around XX kg of CO\textsubscript{2} has been used which equates to X bananas, or X kilometers driven, or X avocados from Mexico.\footnote{\url{https://www.weforum.org/agenda/2020/02/avocado-environment-cost-food-mexico/}} These calculations were made through the CO\textsubscript{2} GU mltgpu tutorial\footnote{\url{https://github.com/faustusdotbe/CO2_GU_mltgpu/blob/main/mltgpu_co2.ipynb}} by Simon Hengchen. These statistics are incredibly important to disclose considering the vast environmental impact conducting these experiments will have. 
\citet{strubell-etal-2019-energy} states that the most computationally-hungry models typically obtain the highest scores. \citet{strubell-etal-2019-energy} also raises awareness on the CO\textsubscript{2} consumption within the NLP community and the importance of reporting of train times and sensitivity to hyperparameters. By providing these statistics, the overall scores, and other files that require computational time and energy consumption, others are able to  further explore and examine this task without having to do all of the model training again. The reporting and analysing of hyperparameter sensitivities and patterns between the models will also allow researchers to apply the intuition when selecting hyperparameters in order to train models that are efficient, environmentally conscious, and yield the best results.



\newpage

\section{Results and Discussion}

\subsection{Overall Top Models (WORDING)}
Specs again?

Hyperparameter searches are purely experimental and can be infinitely expanded. Through using the most often used hyperparameter values in LSC tasks, a foundation is made for where to begin. \autoref{fig:score-dist} shows the distribution of scores for English, German, Latin, and Swedish respectively. The wide variations of scores for each language in these distributions demonstrate the lack of understanding in the effects of the interactions of hyperparameters on the overall performance of the models. (ELABORATE MORE?) A general analysis of the overall top-performing models for each language will be analysed in this section. Findings of the effects of each hyperparameter on the performance for each language will also be discussed below. 

\begin{figure}[h]
  \centering
  \subcaptionbox{English}{\includegraphics[width = 2in]{sections/figures/eng_score_distribution.png}}\quad
  \subcaptionbox{German}{\includegraphics[width = 2in]{sections/figures/deu_score_distribution.png}}\\
  \subcaptionbox{Latin}{\includegraphics[width = 2in]{sections/figures/lat_score_distribution.png}}\quad
  \subcaptionbox{Swedish}{\includegraphics[width = 2in]{sections/figures/sve_score_distribution.png}}
  \caption{Distribution of $\rho$ Scores for Each Language}
  \label{fig:score-dist}
\end{figure}


The hyperparameter search successfully found model combinations that outperformed both the best teams in the SemEval task and \citet{kaiser-etal-2020-ims}’s post-hoc optimizations for both English and Latin. As seen in \autoref{tab:performance-comparison} a model was found to surpass the score of the best team in SemEval but not \citet{kaiser-etal-2020-ims} for the German language. The top-performing models and their hyperparameters are shown in \autoref{tab:top-models}. Although it is important to analyse the top results for each language separately, it is interesting to see commonalities between all four languages. The Orthogonal Procrustes alignment method seems to be the most effective for bigger datasets. For smaller datasets such as Latin, the incremental training alignment method is the most effective as the second time period model has information learned from both the first and second time period corpora. Epochs were also surprisingly low (WHY?) and lower frequency thresholds performed best as it included the maximum amount of words in the model’s vocabulary. 

\begin{table}[h]
\centering
\begin{tabular}{|l|l|l|l|} 
\cline{2-4}
\multicolumn{1}{l|}{\textbf{ }} & SemEval         & Kaiser                   & Best                      \\ 
\hline
English                         & \textit{ .422 } & \textit{ .460 }          & \textbf{\textit{ .469 }}  \\ 
\hline
German                          & \textit{ .725 } & \textit{\textbf{ .780 }} & \textit{ .706 }           \\ 
\hline
Latin                           & \textit{ .462 } & \textit{ .410 }          & \textbf{\textit{ .529 }}  \\ 
\hline
Swedish                         & \textit{ .604 } & \textit{\textbf{ .670 }} & \textit{ .651 }           \\
\hline
\end{tabular}
\caption{Top $\rho$ score comparison between SemEval-2020 Task 2 teams, \citet{kaiser-etal-2020-ims}, and results.}
\label{tab:performance-comparison}
\end{table}


\begin{table}[h]
\centering
\begin{tabular}{cccccccc} 
\toprule
\textbf{ } & \textbf{ Algorithm } & \textbf{ Alignment } & \textbf{ Vocab Size } & \textbf{ Epochs } & \textbf{ Dims } & \textbf{ Freq Threshold } & \textbf{ $\rho$ }  \\
English    & FT              & OP               & ALL (16429)      & 5                 & 300             & 10               & .469            \\
German     & W2V             & OP               & ALL (218507)     & 5                 & 25              & 5                & .706            \\
Latin      & W2V             & INC              & -                & 5                 & 10              & 10               & .529            \\
Swedish    & FT              & OP               & 5000             & 10                & 50              & 10               & .651            \\
\bottomrule
\end{tabular}
\caption{Top-performing models for each language and their parameters. WV=Word2Vec, FT=FastText, OP=Orthogonal Procrustes, INC=Incremental}
\label{tab:top-models}
\end{table}

The following analyses will show the differences in scores when the top-performing model is taken for each language and each hyperparameter is kept the same except for one hyperparameter. For example, Latin’s best model has the hyperparameters: Word2Vec, Incremental Training, 5 epochs, 10 dimensions, 10 frequency threshold. To explore the effects of the change in epochs, all the hyperparameters will remain the same except for epochs. The scores for each of these models are represented by a point in the graph where the $x$-axis is hyperparameter in question and the $y$-axis is  the Spearman $\rho$ score. These points are joined by a dotted line graph and each line represents one of the four languages. Points that are in grey are statistically insignificant where p-value > 0.05 but they are included for X (WHAT TO SAY?). Individual graphs for each language are available to examine in the APPENDIX. (WORDING) In order to extrapolate information from the data collected, the dotted lines are created in order to see trends between the points. However, it is acknowledged that if models were actually trained for each increase in the hyperparameter in question, trends in reality could show very different results. (NECESSARY BUT IS IT ALSO INVALIDATING?)

The relationship between epochs and performance raises the question of priority of performance and environmental impact. In \autoref{fig:all-epochs}, English and Swedish do not have 100-epoch models since it was deemed that their improved performance return was minimal compared to the time spent and energy consumption for training. There is a general decline in $\rho$ as epochs are increased for English, Latin, and Swedish while German had a slight increase in $\rho$ from 50 to 100 epochs. German and Latin appear to experience a similar decrease by 20 epochs but diverge after 50 epochs where German increases and Latin significantly decreases. This divergence might be attributed to the dataset size since the German corpora is much larger and there is more for the model to learn as epochs are increased. On the other hand, the Latin corpora is very limited and after 50 epochs, the decrease in $\rho$ could be additional noise being introduced to the model. Although there might still be an increase in performance when training for more than 50 or 100 epochs, lower epochs seem to be the better choice for LSC. It also is a question of priority—smaller epochs or spending more time, memory, and energy. For this hyperparameter, the option to have a smaller environmental impact compromises very little in terms of performance. 

\begin{figure}[h]
  \centering
  \includegraphics[width=.8\linewidth]{sections/figures/epochs_all.png}
  \caption{Change in Epochs vs. Score}
  \label{fig:all-epochs}
\end{figure}

The dimensions hyperparameter refers to the size of the vector for each word representation. Increasing vector dimensions requires a higher computational time and more memory. As expected, English performs best with higher dimensions. The 300-dimension hyperparameter is commonly used for LSC tasks (CITE?) and is usually the baseline when conducting LSC tasks on other languages due to the prevalence in studies being done in English. As seen in \autoref{fig:all-dims}, choosing higher dimensions does not necessarily result in better $\rho$ scores as it does with the English language. German, Swedish, and Latin all have higher $\rho$ scores when lower dimensions are used. Latin’s top model performed best with the lowest dimension of 10. Since the Latin corpora is very small compared to the other three languages, having less abstractions of words in the vector representations might have been more useful for the model and minimised the introduction of noise. The genre makeup of corpora is also put into question as it seems to have an effect on the vector dimensions and therefore affects performance. Swedish and German are both composed of newspaper corpora and so the same style and type of language is used. English has a mixed genre corpora and Latin is somewhere in between. Since Swedish and German models are trained on texts that have their own writing style where constructions of sentences are similar but the words have been used under many contexts, vectors with higher dimensions allow for those nuances to be learned. In comparison with Latin where the corpora consists of a wide range of texts compiled and vectors with lower dimensions allows the model to learn the usages of the words in a broader context. The genres of text being included in the corpora must be taken into consideration before choosing the vector dimensions for a model. The typical baseline of 300 dimensions, based on English results in past LSC tasks, shows not as effective on other languages. (WORDING?)

\begin{figure}[h]
  \centering
  \includegraphics[width=.8\linewidth]{sections/figures/dims_all.png}
  \caption{Change in Dimensions vs. Score}
  \label{fig:all-dims}
\end{figure}

The frequency threshold hyperparameter refers to the minimum number of times a word has to appear in a corpus to be included in the model’s vocabulary. Smaller frequency thresholds will result in a larger vocabulary. Since the vocabulary is larger, memory needed will also be larger as there are more vectors to be stored and learned by the model.  \autoref{fig:all-freqs} illustrates that it is generally better to have a lower frequency threshold in order to ensure that majority of the words in a corpus is being included. German and Swedish do not have 4 points in \autoref{fig:all-freqs} as the $\rho$ scores for those model combinations were NaNs. While German performs best with the lowest frequency threshold of 5, English, Latin, and Swedish all perform better at a slightly higher frequency threshold of 10. A steady decrease in $\rho$ scores are seen as the frequency threshold increases. This is expected since a higher frequency threshold results in less words in the vocabulary and a smaller vocabulary naturally excludes words that contribute to the contextual meaning of a word. 

\begin{figure}[h]
  \centering
  \includegraphics[width=.8\linewidth]{sections/figures/freqs_all.png}
  \caption{Change in Frequency Thresholds vs. Score}
  \label{fig:all-freqs}
\end{figure}

Two approaches were used for the orthogonal procrustes alignment method—Ryan Heuser’s Tutorial\footnote{\url{https://gist.github.com/quadrismegistus/09a93e219a6ffc4f216fb85235535faf}} and VecMap\footnote{\url{https://github.com/artetxem/vecmap}}. The former had a default of including the entire shared vocabulary of the first time period corpus and the second time period corpus during the alignment process. However, the latter included the option to indicate the size of the shared vocabulary during the alignment process. \autoref{fig:all-vocabalign} shows the effect of vocabulary size during the orthogonal procrustes alignment process on $\rho$ scores using VecMap. Latin is not included in this graph since its top-performing model used the incremental training alignment method. English and German seem to be affected similarly from this hyperparameter while Swedish experiences the opposite. German profits when the entire vocabulary (where the vocabulary size is 218, 507) is included in the alignment process and results in the top-performing model. Swedish experiences a similar increase in performance when the vocabulary size is 5000. However, the top-performing model for Swedish has a vocabulary size of 5000 and an increase in vocabulary size shows a decrease in $\rho$ scores. Although English has a significantly smaller corpora size, it also profits from the inclusion of the entire vocabulary (where the vocabulary size is 16, 429) when aligning both time period models. (NEED CRITIQUE OR CONCLUSION)

\begin{figure}[h]
  \centering
  \includegraphics[width=.8\linewidth]{sections/figures/vocabalignment_all.png}
  \caption{Vocabulary Size in Orthogonal Procrustes Alignment vs. Score}
  \label{fig:all-vocabalign}
\end{figure}

\subsection{POS}

Every model trained was also evaluated by creating smaller target word lists based on each word’s POS-tag. For each language’s target word list, words have been manually annotated FOOTNOTE THANKS POS-tagged as either a noun (NN), verb (VB), or adjective (ADJ). The target word list breakdown for each language is seen in \autoref{fig:target-postags}. Nouns are much more prevalent in the target word lists and so conclusions made for nouns are much more concrete (BETTER WORD, reliable? sure? Interpretable? Or conclusions made for nouns are more meaningful on the grounds of…) compared to conclusions made for verbs and adjectives. Control words within each POS-tag are also shown in FIGURE POS-TAGS through a lighter colour respective of its POS-tag. With already a smaller ratio and an addition of control words that have 0 as a cosine distance, a $\rho$ score of 1 can sometimes be more easily attained since it is calculated by rank. For example, if there are only 2 words for A LANGUAGE adjectives and one is a control word, a model’s chances of obtaining a perfect Spearman $\rho$ score of 1 is very high. 


\begin{figure}[h]
  \centering
  \subcaptionbox{English}{\includegraphics[width = 3in]{sections/figures/eng-pos-target-control.png}}\quad
  \subcaptionbox{German}{\includegraphics[width = 3in]{sections/figures/deu-pos-target-control.png}}\\
  \subcaptionbox{Latin}{\includegraphics[width = 3in]{sections/figures/lat-pos-target-control.png}}\quad
  \subcaptionbox{Swedish}{\includegraphics[width = 3in]{sections/figures/sve-pos-target-control.png}}
  \caption{Target Word POS-tag Distribution for Each Language}
  \label{fig:target-postags}
\end{figure}


\newpage

\section{Ethical Considerations}
\label{sec:ethicalcons}

Alongside an experimental research method such as a hyperparameter search, ethical considerations and acknowledgments must be included. The effects of conducting this experiment is critically assessed and evaluated through the lens of environmental and sociological impact. (SOCIOLOGICAL ASSUMPTIONS MADE? NOT REALLY IMPACT) 

Throughout the process of the hyperparameter search, environmental impact was considered at each step. Although making adjustments were made in order to minimise environmental impact, they also create a more efficient and robust pipeline. As discussed in \autoref{exp-ethics} AUTOREF CO2 , the project had a total of X hours of training and computational time spent and X amount of CO\textsubscript{2} consumed. After training all of the model combinations that use the Word2Vec algorithm, results were evaluated and it was decided that not training 160 model combinations that required 100 epochs of training for FastText would not produce results as significant or meaningful. Since FastText aims to create character \emph{n}-gram vector representations, training 100-epoch models would involve a much larger computational time and memory usage compared to the already large models that were created using Word2Vec. Despite acknowledgments and modifications for minimising consumption and efficiency reasons, a significant amount of energy and memory being used is inevitable. To ensure that the energy consumed and the space taken is worthwhile, all of the results and pipeline will be available for public access (see \autoref{app-resources}). The POS-tagged word lists and cosine distances for each model will also be available for further analysis and evaluations. This large-scale hyperparameter search is also well-documented to ensure that it will not have to be done again and future tasks can focus on refining model optimisations. 

Another crucial consideration that must be made is regarding the data and the type of content being used for training models. Humans have written the texts in the corpora being used and humans are inherently biased. Anonymity is also a concern for researchers when using text as data. Most of the historical texts used for this thesis are quite old, spanning from 2nd century B.C. to 2010 with most of the texts coming from 1800-1900s (FORMAT LOOKS WRONG). The question of anonymity and privacy is not a large concern for this thesis. (FEELS LIKE MISSING A SENTENCE) Through training, machines also inevitably learn the biases from these past texts. Biases such as racism, sexism, homophobia and other forms of discrimination towards marginalised groups of people can be seen throughout historical texts. \citet{tripodi-etal-2019-tracing} show that antisemitic language can be detected through language change. It is extremely vital to recognise that these biases can appear in historical texts and to ensure readers that the author (LOL , JUST USE I?) does not endorse, nor agree with these biases. It is incredibly important to acknowledge the inherent bias that exists within historical data and what might be passed on to trained models. As \citet{hengchen-tahmasebi_2021-swedishdiachronic} state, the hope is to “shed light on these representations, as ignoring them would mean they have never existed”. 


\newpage

\section{Limitations}

The common and most effective approach in this field still presents many critiques and limitations. (TOO VAGUE) The method in which corpora are divided into time bins also present complexities within detecting LSC that must be considered \citep{hengchen2021challenges}. By grouping long periods of time as one entity/time bin/THING (??), the slight variation and changes of meaning words undergo are either lost or minimized. A word or sense’s change in meaning does not have one clear trajectory. As with the SemEval Task, we are essentially detecting the average change (EW) of a word within two timeframes. 

\newpage

\section{Future Work}
\label{sec:futurework}

There are still many questions that remain unanswered in the study of language change. In detecting LSC, finding methodologies that perform well in their respective domains is a continuous undertaking. Ways to expand the hyperparameter search in different aspects will be discussed in this section.

Models trained on diachronic corpora to detect LSC must also be validated in other ways. Although the top models in this thesis performed well based on SemEval’s diachronic corpora, it would be interesting to take the same models and examine the model’s word representations based on synchronic data. Evaluating the validity of these models not only diachronically would give more insight into what word representations have learned and how accurate they are. This can be done by using a dataset such as \citet{supersim2021}'s SuperSim dataset—a similarity and relatedness test set for Swedish based on expert human judgments. 

An evident expansion of this hyperparameter search would be to use the top-performing hyperparameter combinations on other tasks and datasets to assess their robustness and applicability. Conducting similar hyperparameter searches for other languages would allow for a results comparison to see if the same hyperparameter combinations perform best and persist with languages that have similar structures or belong to the same language family. Of course, it is important to keep in mind that availability of datasets for LSC is very limited and the work will begin with creating these datasets. However, once this work has been done, identifying patterns between languages and their top-performing models would allow for a higher-level examination of hyperparameter combinations. Conclusions will hopefully be possible for meta-languages depending on how languages are related—genetic or phylogenetic, or shared language features (e.g. phonology, morphology, syntax). This would allow for more broad intuitions into what models create reliable word representations depending on the language.

On a smaller scale, an examination of the models that performed the worst during this hyperparameter search would also shed some light on what combinations \emph{not} to use. Analysing the worst model combinations in the same method as done with the top-performing models in Section~\ref{sec:results} might shed some light on what makes models unsuccessful. 



\newpage

\section{Conclusion}

In this thesis, a large-scale hyperparameter search is conducted for SemEval-2020 Task 1. The hyperparameter search consists of exploring the effects of language, algorithm, alignment method, epochs, dimensions, frequency threshold, and shared vocabulary size for the Orthogonal Procrustes alignment method on detecting LSC. The results are then presented and analysed by discussing the trends within changes in hyperparameter variables for each language's top model. Models were also evaluated in their performance for detecting LSC depending on the target word's POS-tag. Results (Section~\ref{sec:results}) show that there is a dramatic decrease in performance improvement rate after 50 epochs. It was also demonstrated that the baseline of 300 dimensions based on past English NLP tasks does not necessarily apply to other languages. This highlights the importance of conducting research in languages other than English and the shortcomings of generalising language through the English language. Trends seen in the results of the dimensions, frequency thresholds, and shared vocabulary hyperparameters were also connected to language, corpus size, and text genres. The findings of these thesis ultimately provides a more concrete foundation and basis during the hyperparameter selection process for future tasks in LSC. 





\addcontentsline{toc}{section}{References}
\bibliography{anthology,katesbib}

\newpage
\appendix
\section{Resources (Links)}
\label{app-resources}

%\cmtKV[inline]{is there a way to make this prettier}

\begin{itemize}

  \item Main Results - \href{https://github.com/kateviloria/Semantic-Change-Thesis/blob/main/results/MAIN_ALL.csv}{https://github.com/kateviloria/Semantic-Change-Thesis/blob/main/results/MAIN\_ALL.csv}
    
  \item POS-Tag Results - \href{https://github.com/kateviloria/Semantic-Change-Thesis/blob/main/results/postag-results/MAIN_POS_RESULTS.csv}{https://github.com/kateviloria/Semantic-Change-Thesis/blob/main/results/postag-results/MAIN\_POS\_RESULTS.csv}
  
  \item POS-Tagged Target Word Lists for SemEval-2020 Task 1 - \href{https://github.com/kateviloria/Semantic-Change-Thesis/tree/main/truth-labels}{https://github.com/kateviloria/Semantic-Change-Thesis/tree/main/truth-labels}
  
  \item Models and Respective Cosine Distances - will be available at \href{https://zenodo.org/record/5256293}{https://zenodo.org/record/5256293}
  
\end{itemize}

\newpage

\section{Top-model Graphs (WORDING)}
\label{app-topmodelgraphs}

Each top-performing model is taken and only one parameter is changed to each possible variable examined in the hyperparameter search. How performance is affected by value changes in one variable is visualised through the graphs below. The figures are grouped by language. 

\begin{figure}[h]
  \centering
  \subcaptionbox{Epochs}{\includegraphics[width = 3in]{sections/figures/top-models/eng_epochs.png}}\quad
  \subcaptionbox{Dimensions}{\includegraphics[width = 3in]{sections/figures/top-models/eng_dims.png}}\\
  \subcaptionbox{Frequency Threshold}{\includegraphics[width = 3in]{sections/figures/top-models/eng_freqs.png}}\quad
  \subcaptionbox{Vocabulary Size}{\includegraphics[width = 3in]{sections/figures/top-models/eng_vocabalign.png}}
  \caption{English Top-Model Graphs (MAKE A MORE MEANINGFUL CAPTION)}
  \label{fig:eng-topmodels}
\end{figure}


\begin{figure}[h]
  \centering
  \subcaptionbox{Epochs}{\includegraphics[width = 3in]{sections/figures/top-models/deu_epochs.png}}\quad
  \subcaptionbox{Dimensions}{\includegraphics[width = 3in]{sections/figures/top-models/deu_dims.png}}\\
  \subcaptionbox{Frequency Threshold}{\includegraphics[width = 3in]{sections/figures/top-models/deu_freqs.png}}\quad
  \subcaptionbox{Vocabulary Size}{\includegraphics[width = 3in]{sections/figures/top-models/deu_vocabalign.png}}
  \caption{German Top-Model Graphs (MAKE A MORE MEANINGFUL CAPTION)}
  \label{fig:deu-topmodels}
\end{figure}


\begin{figure}[h]
  \centering
  \subcaptionbox{Epochs}{\includegraphics[width = 3in]{sections/figures/top-models/lat_epochs.png}}\quad
  \subcaptionbox{Dimensions}{\includegraphics[width = 3in]{sections/figures/top-models/lat_dims.png}}\\
  \subcaptionbox{Frequency Threshold}{\includegraphics[width = 3in]{sections/figures/top-models/lat_freqs.png}}\quad
  \caption{Latin Top-Model Graphs (MAKE A MORE MEANINGFUL CAPTION)}
  \label{fig:lat-topmodels}
\end{figure}


\begin{figure}[h]
  \centering
  \subcaptionbox{Epochs}{\includegraphics[width = 3in]{sections/figures/top-models/sve_epochs.png}}\quad
  \subcaptionbox{Dimensions}{\includegraphics[width = 3in]{sections/figures/top-models/sve_dims.png}}\\
  \subcaptionbox{Frequency Threshold}{\includegraphics[width = 3in]{sections/figures/top-models/sve_freqs.png}}\quad
  \subcaptionbox{Vocabulary Size}{\includegraphics[width = 3in]{sections/figures/top-models/sve_vocabalign.png}}
  \caption{Swedish Top-Model Graphs (MAKE A MORE MEANINGFUL CAPTION)}
  \label{fig:sve-topmodels}
\end{figure}

\newpage

\section{POS-Tag Target Word Lists and Results}
\label{app-postags}


The following 4 tables are the POS-tagged target word lists for English, German, Latin, and Swedish. The English graded truth labels were already POS-tagged when downloaded from SemEval-2020 Task 1. German, Latin, and Swedish were manually annotated by annotators that have linguistic backgrounds. The tables are ranked by highest change score (Cosine Distance) to lowest. A 0 in cosine distance indicates that the word is a control word and did not undergo any LSC. These word lists will be made available to the public.


\begin{table}[h]
\centering
\begin{tabular}{ccc} 
\toprule
Words         & POS & Cosine Distance  \\ 
\midrule
plane         & nn  & 0.882348         \\
tip           & vb  & 0.678899         \\
prop          & nn  & 0.624760         \\
graft         & nn  & 0.553976         \\
record        & nn  & 0.427350         \\
ball          & nn  & 0.409367         \\
stab          & nn  & 0.400590         \\
twist         & nn  & 0.398493         \\
bit           & nn  & 0.306577         \\
head          & nn  & 0.295256         \\
ounce         & nn  & 0.284899         \\
rag           & nn  & 0.276515         \\
player        & nn  & 0.273667         \\
edge          & nn  & 0.260966         \\
land          & nn  & 0.223448         \\
lass          & nn  & 0.212590         \\
pin           & vb  & 0.207212         \\
word          & nn  & 0.179307         \\
stroke        & vb  & 0.176231         \\
circle        & vb  & 0.171087         \\
part          & nn  & 0.161271         \\
donkey        & nn  & 0.160104         \\
gas           & nn  & 0.159570         \\
attack        & nn  & 0.143970         \\
thump         & nn  & 0.142992         \\
face          & nn  & 0.137791         \\
quilt         & nn  & 0.123145         \\
lane          & nn  & 0.103720         \\
bag           & nn  & 0.100364         \\
multitude     & nn  & 0.100364         \\
savage        & nn  & 0.096869         \\
contemplation & nn  & 0.070839         \\
tree          & nn  & 0.070839         \\
relationship  & nn  & 0.056218         \\
fiction       & nn  & 0.020723         \\
risk          & nn  & 0.000000         \\
chairman      & nn  & 0.000000         \\
\bottomrule
\end{tabular}
\caption{English Graded Truth Target WordsOLD}
\label{tab:eng-truthtargetsOLD}
\end{table}


\begin{table}
\centering
\begin{tabular}{ccccc} 
\toprule
\textbf{ Words } & \textbf{POS } & \textbf{ Cosine Distance } & \textbf{ C$_1$ Freq } & \textbf{ C$_2$ Freq }  \\ 
\midrule
plane            & nn            & 0.882348           & 4                  & 12                  \\
tip              & vb            & 0.678899           & 31                 & 285                 \\
prop             & nn            & 0.624760           & 20                 & 110                 \\
graft            & nn            & 0.553976           & 33                 & 18                  \\
record           & nn            & 0.427350           & 253                & 276                 \\
ball             & nn            & 0.409367           & 7                  & 34                  \\
stab             & nn            & 0.400590           & 50                 & 83                  \\
twist            & nn            & 0.398493           & 115                & 332                 \\
bit              & nn            & 0.306577           & 46                 & 456                 \\
head             & nn            & 0.295256           & 150                & 840                 \\
ounce            & nn            & 0.284899           & 0                  & 0                   \\
rag              & nn            & 0.276515           & 2                  & 4                   \\
player           & nn            & 0.273667           & 0                  & 1                   \\
edge             & nn            & 0.260966           & 22                 & 73                  \\
land             & nn            & 0.223448           & 262                & 428                 \\
lass             & nn            & 0.212590           & 0                  & 0                   \\
pin              & vb            & 0.207212           & 79                 & 131                 \\
word             & nn            & 0.179307           & 28                 & 11                  \\
stroke           & vb            & 0.176231           & 182                & 198                 \\
circle           & vb            & 0.171087           & 540                & 528                 \\
part             & nn            & 0.161271           & 979                & 441                 \\
donkey           & nn            & 0.160104           & 0                  & 3                   \\
gas              & nn            & 0.159570           & 0                  & 4                   \\
attack           & nn            & 0.143970           & 248                & 370                 \\
thump            & nn            & 0.142992           & 20                 & 58                  \\
face             & nn            & 0.137791           & 208                & 1131                \\
quilt            & nn            & 0.123145           & 10                 & 10                  \\
lane             & nn            & 0.103720           & 2                  & 3                   \\
bag              & nn            & 0.100364           & 14                 & 22                  \\
multitude        & nn            & 0.100364           & 0                  & 0                   \\
savage           & nn            & 0.096869           & 307                & 76                  \\
contemplation    & nn            & 0.070839           & 0                  & 0                   \\
tree             & nn            & 0.070839           & 2                  & 2                   \\
relationship     & nn            & 0.056218           & 0                  & 0                   \\
fiction          & nn            & 0.020723           & 0                  & 0                   \\
risk             & nn            & 0.000000           & 82                 & 169                 \\
chairman         & nn            & 0.000000           & 5                  & 5                   \\
\bottomrule
\end{tabular}
\caption{English Graded Truth Target Words}
\label{tab:eng-truthtargets}
\end{table}


\begin{table}[h]
\centering
\begin{tabular}{ccc} 
\toprule
Words              & POS & Cosine Distance  \\ 
\midrule
abgebrüht          & adj & 0.832645         \\
Ohrwurm            & nn  & 0.832451         \\
Engpaß             & nn  & 0.819957         \\
abbauen            & vb  & 0.740115         \\
ausspannen         & vb  & 0.706690         \\
Eintagsfliege      & nn  & 0.660060         \\
Knotenpunkt        & nn  & 0.647627         \\
artikulieren       & vb  & 0.615743         \\
abdecken           & vb  & 0.606884         \\
Dynamik            & nn  & 0.578845         \\
Abgesang           & nn  & 0.578548         \\
verbauen           & vb  & 0.578125         \\
Fuß                & nn  & 0.564633         \\
Armenhaus          & nn  & 0.519670         \\
zersetzen          & vb  & 0.505880         \\
Rezeption          & nn  & 0.464989         \\
packen             & vb  & 0.462253         \\
Schmiere           & nn  & 0.437671         \\
Sensation          & nn  & 0.406144         \\
Titel              & nn  & 0.393045         \\
Mißklang           & nn  & 0.379723         \\
Manschette         & nn  & 0.355802         \\
beimischen         & vb  & 0.307359         \\
überspannen        & vb  & 0.252066         \\
vorliegen          & vb  & 0.190266         \\
Entscheidung       & nn  & 0.141681         \\
vorweisen          & vb  & 0.126837         \\
Lyzeum             & nn  & 0.126381         \\
voranstellen       & vb  & 0.124192         \\
Tragfähigkeit      & nn  & 0.114694         \\
Spielball          & nn  & 0.103290         \\
Festspiel          & nn  & 0.100364         \\
Ausnahmegesetz     & nn  & 0.093138         \\
Gesichtsausdruck   & nn  & 0.077318         \\
Tier               & nn  & 0.073466         \\
Naturschönheit     & nn  & 0.071561         \\
vergönnen          & vb  & 0.071197         \\
Frechheit          & nn  & 0.070839         \\
Seminar            & nn  & 0.064486         \\
aufrechterhalten   & vb  & 0.036109         \\
Unentschlossenheit & nn  & 0.000000         \\
Ackergerät         & nn  & 0.000000         \\
Truppenteil        & nn  & 0.000000         \\
Pachtzins          & nn  & 0.000000         \\
Mulatte            & nn  & 0.000000         \\
Einreichung        & nn  & 0.000000         \\
weitgreifend       & adj & 0.000000         \\
Kubikmeter         & nn  & 0.000000         \\
\bottomrule
\end{tabular}
\caption{German Graded Truth Target Words OLD}
\label{tab:deu-truthtargetsOLD}
\end{table}


\begin{table}
\centering
\begin{tabular}{ccccc} 
\toprule
\textbf{ Words }   & \textbf{POS } & \textbf{ Cosine Distance } & \textbf{ C$_1$ Freq } & \textbf{ C$_2$ Freq }  \\ 
\midrule
abgebrüht          & adj           & 0.832645           & 64                 & 106                 \\
Ohrwurm            & nn            & 0.832451           & 39                 & 103                 \\
Engpaß             & nn            & 0.819957           & 82                 & 302                 \\
abbauen            & vb            & 0.740115           & 61                 & 1179                \\
ausspannen         & vb            & 0.706690           & 268                & 129                 \\
Eintagsfliege      & nn            & 0.660060           & 59                 & 137                 \\
Knotenpunkt        & nn            & 0.647627           & 360                & 216                 \\
artikulieren       & vb            & 0.615743           & 59                 & 220                 \\
abdecken           & vb            & 0.606884           & 52                 & 366                 \\
Dynamik            & nn            & 0.578845           & 61                 & 681                 \\
Abgesang           & nn            & 0.578548           & 139                & 113                 \\
verbauen           & vb            & 0.578125           & 68                 & 209                 \\
Fuß                & nn            & 0.564633           & 30817              & 3724                \\
Armenhaus          & nn            & 0.519670           & 101                & 127                 \\
zersetzen          & vb            & 0.505880           & 689                & 212                 \\
Rezeption          & nn            & 0.464989           & 101                & 152                 \\
packen             & vb            & 0.462253           & 1806               & 1397                \\
Schmiere           & nn            & 0.437671           & 108                & 122                 \\
Sensation          & nn            & 0.406144           & 170                & 638                 \\
Titel              & nn            & 0.393045           & 3576               & 8637                \\
Mißklang           & nn            & 0.379723           & 54                 & 107                 \\
Manschette         & nn            & 0.355802           & 78                 & 128                 \\
beimischen         & vb            & 0.307359           & 255                & 120                 \\
überspannen        & vb            & 0.252066           & 153                & 194                 \\
vorliegen          & vb            & 0.190266           & 1888               & 1436                \\
Entscheidung       & nn            & 0.141681           & 4030               & 8582                \\
vorweisen          & vb            & 0.126837           & 68                 & 371                 \\
Lyzeum             & nn            & 0.126381           & 107                & 118                 \\
voranstellen       & vb            & 0.124192           & 156                & 194                 \\
Tragfähigkeit      & nn            & 0.114694           & 82                 & 326                 \\
Spielball          & nn            & 0.103290           & 59                 & 176                 \\
Festspiel          & nn            & 0.100364           & 50                 & 708                 \\
Ausnahmegesetz     & nn            & 0.093138           & 59                 & 170                 \\
Gesichtsausdruck   & nn            & 0.077318           & 101                & 157                 \\
Tier               & nn            & 0.073466           & 30574              & 4335                \\
Naturschönheit     & nn            & 0.071561           & 98                 & 130                 \\
vergönnen          & vb            & 0.071197           & 706                & 205                 \\
Frechheit          & nn            & 0.070839           & 362                & 213                 \\
Seminar            & nn            & 0.064486           & 192                & 1975                \\
aufrechterhalten   & vb            & 0.036109           & 40                 & 1171                \\
Unentschlossenheit & nn            & 0.000000           & 76                 & 130                 \\
Ackergerät         & nn            & 0.000000           & 53                 & 106                 \\
Truppenteil        & nn            & 0.000000           & 169                & 630                 \\
Pachtzins          & nn            & 0.000000           & 39                 & 103                 \\
Mulatte            & nn            & 0.000000           & 139                & 116                 \\
Einreichung        & nn            & 0.000000           & 60                 & 141                 \\
weitgreifend       & adj           & 0.000000           & 64                 & 103                 \\
Kubikmeter         & nn            & 0.000000           & 123                & 1903                \\
\bottomrule
\end{tabular}
\caption{German Graded Truth Target Words}
\label{tab:deu-truthtargets}
\end{table}


\begin{table}[h]
\centering
\begin{tabular}{ccc} 
\toprule
Words       & POS & Cosine Distance  \\ 
\midrule
pontifex    & nn  & 0.905600         \\
imperator   & nn  & 0.846816         \\
beatus      & adj & 0.816392         \\
sacramentum & nn  & 0.688040         \\
titulus     & nn  & 0.619797         \\
potestas    & nn  & 0.548475         \\
scriptura   & nn  & 0.516652         \\
licet       & vb  & 0.506818         \\
salus       & nn  & 0.469503         \\
humanitas   & nn  & 0.455671         \\
sanctus     & adj & 0.425203         \\
virtus      & nn  & 0.397110         \\
sensus      & nn  & 0.393510         \\
credo       & vb  & 0.370992         \\
templum     & nn  & 0.370184         \\
nepos       & nn  & 0.364883         \\
regnum      & nn  & 0.355575         \\
jus         & nn  & 0.350099         \\
adsumo      & vb  & 0.342616         \\
dubius      & adj & 0.337623         \\
civitas     & nn  & 0.322392         \\
honor       & nn  & 0.290373         \\
dux         & nn  & 0.289054         \\
cohors      & nn  & 0.280830         \\
senatus     & nn  & 0.264773         \\
sapientia   & nn  & 0.234681         \\
poena       & nn  & 0.230906         \\
nobilitas   & nn  & 0.181606         \\
dolus       & nn  & 0.176682         \\
fidelis     & adj & 0.170439         \\
acerbus     & adj & 0.169367         \\
voluntas    & nn  & 0.144737         \\
consul      & nn  & 0.129886         \\
consilium   & nn  & 0.102932         \\
oportet     & vb  & 0.102492         \\
necessarius & adj & 0.095190         \\
itero       & vb  & 0.039678         \\
simplex     & adj & 0.009540         \\
hostis      & nn  & 0.000000         \\
ancilla     & nn  & 0.000000         \\
\bottomrule
\end{tabular}
\caption{Latin Graded Truth Target WordsOLD}
\label{tab:lat-truthtargetsOLD}
\end{table}


\begin{table}
\centering
\begin{tabular}{ccccc} 
\toprule
\textbf{ Words } & \textbf{ POS } & \textbf{ Cosine Distance } & \textbf{ C$_1$ Freq } & \textbf{ C$_2$ Freq }  \\ 
\midrule
pontifex         & nn            & 0.905600           & 134                & 1488                \\
imperator        & nn            & 0.846816           & 732                & 10617               \\
beatus           & adj           & 0.816392           & 344                & 2865                \\
sacramentum      & nn            & 0.688040           & 43                 & 1674                \\
titulus          & nn            & 0.619797           & 45                 & 1546                \\
potestas         & nn            & 0.548475           & 913                & 3998                \\
scriptura        & nn            & 0.516652           & 27                 & 1549                \\
licet            & vb            & 0.506818           & 889                & 6043                \\
salus            & nn            & 0.469503           & 671                & 3572                \\
humanitas        & nn            & 0.455671           & 125                & 546                 \\
sanctus          & adj           & 0.425203           & 190                & 10319               \\
virtus           & nn            & 0.397110           & 1643               & 5452                \\
sensus           & nn            & 0.393510           & 431                & 3512                \\
credo            & vb            & 0.370992           & 1660               & 8941                \\
templum          & nn            & 0.370184           & 439                & 3231                \\
nepos            & nn            & 0.364883           & 68                 & 1186                \\
regnum           & nn            & 0.355575           & 700                & 7285                \\
jus              & nn            & 0.350099           & 1209               & 6683                \\
adsumo           & vb            & 0.342616           & 45                 & 338                 \\
dubius           & adj           & 0.337623           & 423                & 2026                \\
civitas          & nn            & 0.322392           & 1693               & 7074                \\
honor            & nn            & 0.290373           & 897                & 4995                \\
dux              & nn            & 0.289054           & 914                & 5128                \\
cohors           & nn            & 0.280830           & 293                & 1032                \\
senatus          & nn            & 0.264773           & 2630               & 2228                \\
sapientia        & nn            & 0.234681           & 306                & 2642                \\
poena            & nn            & 0.230906           & 447                & 3772                \\
nobilitas        & nn            & 0.181606           & 183                & 642                 \\
dolus            & nn            & 0.176682           & 106                & 672                 \\
fidelis          & adj           & 0.170439           & 131                & 2801                \\
acerbus          & adj           & 0.169367           & 149                & 349                 \\
voluntas         & nn            & 0.144737           & 536                & 3214                \\
consul           & nn            & 0.129886           & 3992               & 2513                \\
consilium        & nn            & 0.102932           & 1772               & 4926                \\
oportet          & vb            & 0.102492           & 1278               & 3414                \\
necessarius      & adj           & 0.095190           & 409                & 2978                \\
itero            & vb            & 0.039678           & 32                 & 247                 \\
simplex          & adj           & 0.009540           & 143                & 2141                \\
hostis           & nn            & 0.000000           & 3088               & 6033                \\
ancilla          & nn            & 0.000000           & 96                 & 603                 \\
\bottomrule
\end{tabular}
\caption{Latin Graded Truth Target Words}
\label{tab:lat-truthtargets}
\end{table}


\begin{table}[h]
\centering
\begin{tabular}{ccc} 
\toprule
Words        & POS & Cosine Distance  \\ 
\midrule
medium       & nn  & 0.603554         \\
krita        & nn  & 0.442764         \\
motiv        & nn  & 0.353028         \\
ledning      & nn  & 0.337868         \\
granskare    & nn  & 0.319612         \\
uppläggning  & nn  & 0.293040         \\
beredning    & nn  & 0.263710         \\
konduktör    & nn  & 0.247635         \\
bearbeta     & vb  & 0.243557         \\
notis        & nn  & 0.212213         \\
uppfattning  & nn  & 0.195435         \\
undertrycka  & vb  & 0.166747         \\
färg         & nn  & 0.163421         \\
blockera     & vb  & 0.159570         \\
antyda       & vb  & 0.143225         \\
central      & adj & 0.122427         \\
kemisk       & adj & 0.100908         \\
aktiv        & adj & 0.087075         \\
by           & nn  & 0.074686         \\
gagn         & nn  & 0.071197         \\
annandag     & nn  & 0.070839         \\
bröllop      & nn  & 0.070839         \\
studie       & nn  & 0.070443         \\
förhandling  & nn  & 0.002218         \\
bedömande    & vb  & 0.000443         \\
kokärt       & nn  & 0.000000         \\
bolagsstämma & nn  & 0.000000         \\
uppfostran   & nn  & 0.000000         \\
uträtta      & vb  & 0.000000         \\
vaktmästare  & nn  & 0.000000         \\
vegetation   & nn  & 0.000000         \\
\bottomrule
\end{tabular}
\caption{Swedish Graded Truth Target WordsOLD}
\label{tab:sve-truthtargetsOLD}
\end{table}


\begin{table}
\centering
\begin{tabular}{ccccc} 
\toprule
\textbf{ Words } & \textbf{ POS } & \textbf{ Cosine Distance } & \textbf{ C$_1$ Freq } & \textbf{ C$_2$ Freq }  \\ 
\midrule
medium           & nn            & 0.603554           & 706                & 517                 \\
krita            & nn            & 0.442764           & 473                & 377                 \\
motiv            & nn            & 0.353028           & 147                & 2719                \\
ledning          & nn            & 0.337868           & 1373               & 13459               \\
granskare        & nn            & 0.319612           & 209                & 89                  \\
uppläggning      & nn            & 0.293040           & 164                & 199                 \\
beredning        & nn            & 0.263710           & 765                & 2095                \\
konduktör        & nn            & 0.247635           & 104                & 1897                \\
bearbeta         & vb            & 0.243557           & 249                & 1645                \\
notis            & nn            & 0.212213           & 241                & 4705                \\
uppfattning      & nn            & 0.195435           & 83                 & 9044                \\
undertrycka      & vb            & 0.166747           & 254                & 1629                \\
färg             & nn            & 0.163421           & 7087               & 15036               \\
blockera         & vb            & 0.159570           & 736                & 235                 \\
antyda           & vb            & 0.143225           & 3103               & 3685                \\
central          & adj           & 0.122427           & 123                & 4012                \\
kemisk           & adj           & 0.100908           & 108                & 3926                \\
aktiv            & adj           & 0.087075           & 121                & 1461                \\
by               & nn            & 0.074686           & 6953               & 15499               \\
gagn             & nn            & 0.071197           & 899                & 3717                \\
annandag         & nn            & 0.070839           & 751                & 2031                \\
bröllop          & nn            & 0.070839           & 255                & 4702                \\
studie           & nn            & 0.070443           & 906                & 3744                \\
förhandling      & nn            & 0.002218           & 95                 & 9122                \\
bedömande        & vb            & 0.000443           & 130                & 1478                \\
kokärt           & nn            & 0.000000           & 167                & 203                 \\
bolagsstämma     & nn            & 0.000000           & 1270               & 12516               \\
uppfostran       & nn            & 0.000000           & 5066               & 3413                \\
uträtta          & vb            & 0.000000           & 2841               & 3746                \\
vaktmästare      & nn            & 0.000000           & 162                & 2693                \\
vegetation       & nn            & 0.000000           & 156                & 313                 \\
\bottomrule
\end{tabular}
\caption{Swedish Graded Truth Target Words}
\label{tab:sve-truthtargets}
\end{table}


% ----------------
% The tables below show the top-performing models based on the smaller target word lists created by sorting the target words by POS-tag. For each table, the first row is the overall top-performing model (entire target word list) which is included for easier comparisons. 










\newpage

\end{document}


