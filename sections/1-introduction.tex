
\section{Introduction}
\label{sec:intro}

\subsection{1.X - Motivation}
As with many approaches where linguistics and computational methods intersect, “computational models of word meaning are often taken at face value and not questioned by researchers working on LSC” \citep{hengchen2021challenges}. Evaluating models not only in their performance but also grounded with reasoning based on linguistic theory is crucial. Since models are language-specific and time-specific, assessments in the field of LSC must also be analyzed with linguistic and sociological lenses. Through conducting a hyperparameter search, all other variables are controlled and models can be thoroughly evaluated and analyzed through the two lenses mentioned above. 

\citet{hengchen2021challenges} also states that there, understandably so due to the large availability and resources put into creating English corpora, has been a large over-representation of studies performed on English. Although there have been great advances for the English language in this field, these tools and methods are not necessarily transferable or applicable to other languages. By using the SemEval Dataset, more languages other than English are being represented and considered (WEIRD).  

Advancements in the field of LSC would be beneficial to many other fields and real-world applications. LSC approaches can be used by lexicographers in identifying and validating the usages and dates of usage of specific word senses (ANNOTATIONS FOR XAMPLES). The field of historical linguistics would also benefit from these approaches in order to test the different laws or hypotheses involving how languages change (ANNOTATIONS FOR EXAMPLES) It is also mentioned in \citet{hengchen2021challenges} that these methods are transferable to other fields such as the interpretation of literature in historical research and political science.  
\subsection{1.X - Scope and Limitations}
Distributional approaches face many criticisms in the field of computational semantics. Through an evaluation of their novel approach that considers syntactic relations and traditional vector-based models, \citet{pado-lapata-2003-constructing} note that traditional semantic space models have a more difficult time differentiating semantic relations between word pairs. Models used today for detecting semantic change still have a difficult time discerning the type of semantic relation or change that is occurring in a vector space. Evaluating the specification of semantic relations and changes between words is not within the scope of this thesis. Performance of the models are based on whether or not it can accurately detect whether a word has undergone semantic change. The ability to discern whether or not a target word has gained or lost a sense is not evaluated. 

Previous work \citep{hengchen2021SBXrushifteval} blablabla 

Testing Unicode: Göteborgs universitet

\textit{Testing} \textbf{testing} \textsc{testing} some font series.

Testing a formula:
\[
P(X) = \sum_{i=1}^N P(A_i) P(X|A_i)
\]

Testing a table:
\begin{table}[htbp]
\begin{center}
\begin{tabular}{c|c}
cell 1 & cell 2 \\
\hline
cell 3 & cell 4
\end{tabular}
\caption{This is a table.}
\end{center}
\end{table}