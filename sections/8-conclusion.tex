\section{Conclusion}

In this thesis, a large-scale hyperparameter search is conducted for SemEval-2020 Task 1. The hyperparameter search consists of exploring the effects of language, algorithm, alignment method, epochs, dimensions, frequency threshold, and shared vocabulary size for the Orthogonal Procrustes alignment method on detecting LSC. The results are then presented and analysed by discussing the trends within changes in hyperparameter variables for each language's top model. Models were also evaluated in their performance for detecting LSC depending on the target word's POS-tag. Results (Section~\ref{sec:results}) show that there is a dramatic decrease in performance improvement rate after 50 epochs. It was also demonstrated that the baseline of 300 dimensions based on past English NLP tasks does not necessarily apply to other languages. This highlights the importance of conducting research in languages other than English and the shortcomings of generalising language through the English language. Trends seen in the results of the dimensions, frequency thresholds, and shared vocabulary hyperparameters were also connected to language, corpus size, and text genres. The findings of these thesis ultimately provides a more concrete foundation and basis during the hyperparameter selection process for future tasks in LSC. 