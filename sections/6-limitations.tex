\section{Critiques and Limitations}
\label{sec:critiques}

The common and most effective approach in detecting language change still presents limitations and makes many assumptions that must be acknowledged and discussed. This section will discuss components of semantic change that are overlooked with the approach used for this thesis and its shortcomings. 

As with any task, different approaches have different aims and results. When detecting LSC through vector representations, the choice of what each vector represents can significantly change the results. For this thesis, each word in the vocabulary is represented by one vector only. However, by doing so the meaning of a word is limited, or arguably broadened. Words that have more than one sense or meaning will only be represented by an average of all of the times the word has been used in the corpus. This word vector representation can then be affected by the frequencies of the senses. If one sense appears more in a corpus than the other, its representation will skew towards the sense that is more prevalent.  When LSC is detected, the word’s multiple senses will be considered one entity. For example, if both of a word’s senses such as ‘arms’ as in the body part and ‘arms’ as in weapons appear in the same corpus, it will result in having one vector representation although each sense is different from the other. Using this approach while also using corpora that have time period spans as wide as multiple centuries can also overlook and skew the data. \citet{shoemark-etal-2019-room} show that there is a high chance that seasonal changes in words interfere with detecting semantic change within short time periods and present synthetic evaluation as a solution. An example of a seasonal trend would be the drastic change in frequency and use of the word 'turkey' during Thanksgiving. It is also important to acknowledge that this approach is inherently biased towards word frequency and lead to noise being introduced within the embeddings \citep{dubossarsky-etal-2017-outta, kaiser-etal-2020-ims, schlechtweg-etal-2020-semeval}.

Pre-processing of texts can also have a large effect on the scale and types of semantic change that can be detected. Methods such as lemmatization, lowercasing words, and removal of punctuation can limit what can be detected in terms of language change. For example, the change of 'apple' the fruit to 'Apple' the company would be difficult to detect if all use of the words in a corpus were lowercased \citep{tahmasebi-survey2018}. Since the SemEval corpora was gathered for a specific task with target words, this is not a great concern. However, it is something that must be considered when models are trained for general observation of semantic change.

The method in which corpora are divided into time periods also present complexities within detecting LSC that must be considered \citep{hengchen2021challenges}. By grouping long periods of time as one time period, the slight variation and changes of meaning words undergo are either lost or minimized. A word or sense’s change in meaning does not have one clear trajectory. As with the SemEval Task, we are essentially detecting the average change of a word within two time periods. In order to detect “smaller” semantic changes, the time periods and the size of the corpora must be reduced. The genres of text within the corpora must also be considered. The texts used for English, German, and Swedish all contain newspaper corpora from specific time periods. It is important to be mindful that this kind of corpora is representative of a specific part of culture and society and is composed of a specific type of writing style. These corpora used with the Distributional Hypothesis “often conflate lexical meaning with cultural and topical information available in the corpus used as a basis for the model” \citep{hengchen2021challenges}. With such limited contexts, a question to keep in mind is whether newspaper corpora provide enough insight into the usages of a word and therefore the changes a word undergoes \citep{hengchen2021challenges}. Using a mix of text genres will provide a wider variety of word usages as one word sense could be used more in one genre over the other. Another solution would be to use textual genres and other social factors as features for the words' vector representations \citep{perrone-etal-2019-gasc, jawahar-seddah-2019-contextualized}. For SemEval, this is mitigated by having annotators provide the ground truth of a limited sample of a language using the DURel framework \citep{DURel2018} instead of referring to a dictionary. (MISSING EXPLAINING SENTENCE?, FEELS DISJOINTED) In detecting LSC, as is the case with other research areas, domain-specific tasks will require different datasets and different methods of processing data. There is no one method to achieve all of the tasks of language change and different types of change in meaning can be detected through different methodologies. 
