\section{Background}
% subsections are \subsection{title}
%% subssubsections are \subsubsection{title}
%% numbering will work automatically
\subsection{1.1 - Distributional Hypothesis}

The Distributional Hypothesis theoretically drives the current and leading models for detecting Lexical Semantic Change (LSC). The rationale being that “there is a correlation between distributional similarity and meaning similarity, which allows us to utilize the former in order to estimate the latter”—in simpler and more familiar terms, “words which are similar in meaning occur in similar contexts.” \citep{sahlgren2008distributional} The distributional methodology presented in \citet{harris1970distributional} is built on structuralist theory. A structuralist approach to language focuses on the general construction of a language system rather than the idiolectal use of language. According to \citet{sahlgren2008distributional}, Saussure identifies the functional differences of linguistic meaning into syntagmatic and paradigmatic relations. Syntagmatic relations involve the syntactic positioning or sequence of words. The combination and order of linguistic entities form a syntagmatic relationship that then creates meaning. Paradigmatic relations are between words that appear in the same context but do not co-occur. Given these characteristics, linguistic entities that have a paradigmatic relationship should be interchangeable within the same context or sentence. 