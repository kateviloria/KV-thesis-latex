\section{Background}
\label{sec:background}
% subsections are \subsection{title}
%% subssubsections are \subsubsection{title}
%% numbering will work automatically

\subsection{2.1 - Distributional Hypothesis}

The Distributional Hypothesis is the theory that drives the current and leading models for detecting LSC. The rationale being that “there is a correlation between distributional similarity and meaning similarity, which allows us to utilize the former in order to estimate the latter”—in simpler and more familiar terms, “words which are similar in meaning occur in similar contexts.” \citep{sahlgren2008distributional} The distributional methodology presented in \citet{harris1970distributional} is built on structuralist theory. A structuralist approach to language focuses on the general construction of a language system rather than the idiolectal use of language. According to \citet{sahlgren2008distributional}, Saussure identifies the functional differences of linguistic meaning into syntagmatic and paradigmatic relations. Syntagmatic relations involve the syntactic positioning or sequence of words. The combination and order of linguistic entities form a syntagmatic relationship that then creates meaning. Paradigmatic relations are between words that appear in the same context but do not co-occur. Given these characteristics, linguistic entities that have a paradigmatic relationship should be interchangeable within the same context or sentence. \citet{sahlgren2008distributional} offers the refined distributional hypothesis as “A distributional model accumulated from co-occurrence information contains syntagmatic relations between words, while a distributional model accumulated from information about shared neighbors contains paradigmatic relations between words.” With this in mind, models with a larger context window are more likely to detect or learn paradigmatic relations. Depending on the manipulation of specific model hyperparameters, certain relations will be learned. 


\subsection{2.X - Lexical Semantic Change(??)}

LSC detection through computational methods still face many challenges today. \citet{hengchen2021challenges} have identified two approaches in the computational field of LSC—treating a word as an entity and determining semantic change based on its dominant sense and treating each word’s sense as a separate entity. Both approaches however, mostly capture contextual similarity between lexical items while the different levels of meaning are seldom distinguished \citep{hengchen2021challenges}. There are three different methods to model the meaning of a word computationally: each word in the vocabulary and all its semantic information will have one representation (e.g., static embeddings), each word being split into different semantic areas resulting in representations that approximate a word’s senses (e.g., topic modelling), and each word having a representation for every time it is used in a sentence (e.g., contextual embedding). Each method can be successful given it is applied appropriately in different tasks and what kind of LSC problem is being solved. It is also important to note that not all approaches are able to model or differentiate the different senses a word could possess. In addition to the word sense limitations these representations have, count representations have also been proven to introduce an inherent dependence on word frequency—resulting in random noise in the models \citep{dubossarsky-etal-2017-outta}. By comparing different corpora with each other, original and shuffled, \citet{dubossarsky-etal-2017-outta} demonstrates that there is a strong correlation between the change scores of words and their frequencies. 

Once words have been created into representations, there are also different calculations that can be used when comparing word representations between two time periods (e.g. cosine distance, Euclidean distance, Jensen-Shannon divergence, etc.). With these calculations, there are two ways to evaluate large datasets and tasks involving the detection of LSC—binary LSC (i.e. has a word changed or not) or a graded LSC (i.e. to what degree has a word changed). Although these approaches present a systematic way of evaluating the current models of meaning being created, the question of what kind of change in meaning has a word undergone remains unanswered.

Corpora and datasets used in the field of LSC are largely in English. However, to assume that semantic change occurs in similar ways across other languages is extremely incorrect. As stated by \citet{bender_2020}, the advancements in NLP (ACRONYM) rely on the existence of language resources and English is neither synonymous with nor representative of “Natural Language”. English, as a high resource language, naturally results in more research being published on English. The field of LSC is no different. Apart from a handful of corpora in different languages (\citet{diacrita_evalita2020} for Italian and \citet{rushifteval2021} for Russian), most research in LSC is conducted on the English language. In order to study semantic change effectively and successfully, the variation that already exists between languages must be considered. The creation of resources for languages other than English is crucial to the development of LSC and in turn (NOT USING RIGHT?), NLP.   

\subsubsection{Laws of Semantic Change}
Unlike the laws of sound change, the laws of semantic change are not as developed and theories supporting these laws have not been developed extensively. \citep{Xu2015ACE} \citet{Xu2015ACE} is one study where two opposing laws are examined and evaluated using computational methods and large historical corpora. The law of differentiation is the tendency that words that are near-synonyms have a tendency to diverge or differentiate in meaning as time progresses while the law of parallel change states that words that have related meanings will change in meaning in the same way. (\citet{breal1897essai} and \citet{stern-1921} respectively) \cmtKV{is this citation allowed SH: Yes! I'd rather have you cite too much than not enough} \citet{Xu2015ACE} show that the law of parallel change is more prevalent than the law of differentiation in the corpora they have examined. Continued research of semantic change through computational methods allows theories to be confirmed, disproved, or improved upon. 

Extending \citet{hamilton-etal-2016-diachronic}’s proposed method of formulating statistical laws of semantic change through distributional methods in one language, \citet{uban-etal-2019-studying} examine the same laws cross-lingually by calculating the semantic distance of cognate words in numerous languages. Cognates are words in languages from the same language family that share a proto-word. After differentiating cognates and false friends within the corpora, it is shown that “the frequency and polysemy of cognates positively correlate with their cross-lingual semantic shift”. \citep{uban-etal-2019-studying}

* VERBS CHANGE MORE THAN NOUNS, DUBOSSARSKY \cmtKV[inline]{unsure how to connect this to this section, but i remember us discussing that this should be brought up}
\cmtSH[inline]{This was brought up when we discussed that you, unlike the majority of the work before you, looked at different POS-tags. While you do not, in your thesis, investigate whether verbs change more than nouns or whether verbs change more often than nouns or ... etc., you go beyond the pure abstract concept of ``a word is a word" and look at different features. I thought it would be interesting to reflect on that -- you challenge, following a few others, the current status quo}

\subsubsection{Context Region}
\cmtKV[inline]{i'm also unsure what and how much to say about this and again relating it back to this section}

\cmtSH[inline]{This was brought up when we discussed the work of Paul Nulty: he has tried to keep the 5-word window, but having that 5-word window much further away on both sides, with the aim of getting more paradigmatic relationships than syntagmatic ones. This can be mentioned in passing when discussing the ``context window" hyperparameter}

\subsection{2.X - Algorithms and Architectures (RETHINK HEADING?)}
The two main algorithms used to create word vector representations in this thesis, Word2Vec and FastText, are introduced below. They share similar approaches but also consider different aspects of language. (BLEH IS AN INTRO NEEDED HERE?)

\cmtSH[inline]{Yes I believe a short intro can fit nicely. It will allow you to write a few words about vector space, about how those ``predicting vectors" are ``better" than the count vectors IN NLP IN GENERAL (https://aclanthology.org/P14-1023/ !), and hammer the nail on the fact that until VERY recently the type embeddings were the stars of LSC. Titles of papers such as ``OP-IMS@ DIACR-Ita: Back to the Roots: SGNS+ OP+ CD still rocks Semantic Change Detection" (Kaiser et al 2020) and ``CL-IMS@ DIACR-Ita: Volente o Nolente: BERT does not outperform SGNS on Semantic Change Detection" (Laicher et al 2020) could be mentioned here for example, as a strong reminder of your motivation of using those algos and not BERT.} 

\subsubsection{Word2Vec}
\citet{mikolov2013efficient} proposed two new architectures that use simpler neural networks to learn distributed representations of words while minimizing computational complexity. This resulted in the widely used Word2Vec model architecture that consists of two algorithms—Continuous Bag-of-Words (CBOW) and Continuous Skip-gram. Both algorithms use a word’s contextual environment (i.e. its surrounding words or context words) to create its semantic representation, in this case an embedding. The number of neighbouring words before and after (i.e. window size) for training examples is a hyperparameter that can be changed. The three main components of the Word2Vec architecture are a vocabulary builder, a context builder, and a neural network with two layers. The first component’s input is raw text from a corpora and it creates a vocabulary containing all of the unique words and the number of occurrences for each word within the corpora. The context builder uses the output of the vocabulary builder to create word pairings before and after the target word depending on the context size. Finally, the neural network consists of an input layer, a hidden layer, and an output layer which is then put through a softmax classifier to convert the output (LOGITS?) into probabilities. Since the vector’s shape is the size of the vocabulary, the final vector contains the probability of each word appearing beside the target word. 

In the CBOW model, the target word is predicted through the distributional representations of the context words. However, in Skip-gram the input and output are the inverse of CBOW—the context words are predicted using the target word.  Skip-gram performs well when used on smaller datasets and have better representations of less frequent words while CBOW’s models train faster and have better representations of more frequent words \citep{mikolov2013efficient}. Both algorithms were evaluated through performing basic algebraic equations using the vector representations of words to substantiate the models’ ability to identify both semantic and syntactic relationships between words. This seminal paper is now the basis for tasks such as dependency parsing, machine translation, sentiment analysis, and named entity recognition.


In \citet{mikolov2013distributed}, extensions to the continuous Skip-gram model that decrease computational time and improve the vector representations are introduced. The subsampling approach, called skipgram with negative sampling (SGNS), takes into consideration that high frequency words, such as “the” or “in”, sometimes does not contribute much information on the meaning of a word compared to less frequent words. By removing specific instances of a high-frequency word, there would be less training examples resulting in a decrease in computational time and an improved “accuracy of the representations of less frequent words”. \citep{mikolov2013distributed} The probability that a word is removed is based on its frequency within the entire training text. The second extension presented is negative sampling which is an alternative to hierarchical softmax and Noise Contrastive Estimation (NCE). Instead of adjusting all of the weights within the skip-gram’s neural network for every training sample, each sample only modifies some. Depending on n, only n words whose output should be 0 (labels in the network are one-hot vectors) will have their weights updated. (WORDING) The weights for the target word whose output should be 1 will also be updated. Adjusting a much smaller percentage of the weights in the output layer also helps decrease computational time.

\subsubsection{FastText}
Another extension of the skipgram model resulting in a new approach for creating word representations is the FastText algorithm introduced in \citet{bojanowski2017enriching}. This algorithm was proposed on the basis that the skipgram model does not take into consideration subword information. The morphology or structure of the words are not being learned by the model and this could significantly decrease a model’s performance—especially for morphologically-rich languages. Representations are learned for character \emph{n}-grams and words are the sum of the \emph{n}-gram vectors.\cmtSH{Here, perhaps an example to illustrate? You could have ENG ``the boy" \texttt{the} \texttt{boy} vs SE \texttt{pojken}, where in FastText \texttt{pojken} is very close to \texttt{pojke} in Swedish but not eg in English word2vec?} The bag of character \emph{n}-gram for each word also includes the word itself. Character level information also allows better, more reliable representations for out-of-vocabulary words and is more robust when using bigger corpora for training. Conversely (HMM?), FastText still performs relatively well when given a smaller training dataset since it is able to create reliable representations of out-of-vocabulary words. Through this approach, \citet{bojanowski2017enriching} was able to demonstrate that specific n-grams in a word can correspond to morphemes and in turn is more accommodating to languages such as German that largely consists of noun compounding. 
\subsection{2.X - Alignment Methods}
\cmtKV[inline]{is this section elaborate enough? I'm not sure if I should be speaking in a more technical sense or if this is enough}
\cmtSH[inline]{I think you should start even lower -- WHY do you need to align? For many it's not clear as to why they need to, and in some fields (like DH) they don't align, leading to papers published with factually wrong info (since reviewers also don't know there's a need for alignment). 

Don't forget to cite Kim et al (incremental training) -- this is actually the first time people ``aligned". Who aligned first with OP? Are there other methods of alignment (answer: temporal referencing), etc.? Have a look at that section in the survey by Nina et al for inspiration and factual information}

When detecting LSC, there are many ways to align diachronic corpora. The two common and best-performing techniques, Orthogonal Procrustes and incremental training, are discussed below. (AGAIN IS SOMETHING LIKE THIS NEEDED?) Alignment techniques are required for static neural embeddings in order to compare vector representations of a word between two time periods. 

%\subsubsection{Incremental Training}
Incremental training is an alignment method that makes use of corpora from both time periods and is ideal for languages that have much smaller datasets. It requires training and saving a model using the corpus from the first time period. Then, taking this same model and training it on the corpus from the second time period using the exact same hyperparameters as the first. The two models are then used to compare the vector representations of the target words. 

%\subsubsection{Orthogonal Procrustes}
Aligning through Orthogonal Procrustes differs from incremental training as each corpus is trained separately using the same hyperparameters. With the two models, one is selected to be the basis of the two—in this case, it is the model trained on the corpus from the first time period. The procrustes analysis finds the optimal orthogonal linear transformation of the second model with respect to the first model. In simpler terms, the second model’s vector space and the words within that vector space are stretched and transformed to fit and align with the first model’s vector space.